\documentclass{article}
\usepackage[english]{babel}
\usepackage[utf8]{inputenc}
\usepackage{fancyhdr}
\usepackage{geometry}
 \geometry{
 a4paper,
 left=30mm,
 top=30mm,
 }

%Titlepage
\thispagestyle{fancy}
\fancyhf{}
\setlength{\headheight}{22.54448pt}
\rhead{University of Sussex - Informatics\\
Computer Science with Artificial Intelligence}
\lhead{Jacob Brown}
\rfoot{
Submission Year - 2022\\
Candidate Number - 198732\\
Project Supervisor - Simon Bowes\\
}

\begin{document}
\paragraph*{
\\
}
\part*{
\begin{center}
{ \Huge "Lizardbot"}
\\[1\baselineskip]
{\Large A reptile-inspired model of a robot optimised for navigating rough terrain}
\end{center}
}
\paragraph*{Abstract\\}
Insert abstract here
\vspace*{\fill}
\newpage
\pagestyle{fancy}
\fancyhf{}
\rhead{PAGE \thepage}
\lhead{LIZARDBOT - \leftmark}

\tableofcontents

%Report
\newpage
\section{Introduction}
Add in the intro pretty much directly from the interim report here

\newpage
\section{Project Aims}
Why is the robot being modelled instead of physically built?\\
Why did I choose to use Unity?\\
\subsection{Primary Objectives}
\subsubsection{Robot Design}
Get from interim report - explain overall design and why those decisions were made e.g. simplistic design
\subsubsection{Robot Movement}
Basic overview of why each component will move the way it does. Tie each point back to how they are founded (or not founded) in natural algorithms.\\ 
Include jumping here\\
\subsubsection{Terrain Generation}
The terrain will be static - why?\\
Why will I have three separate terrains? - Octopus\\
What is the importance of having a smooth terrain?\\
\subsubsection{AI}
How will the AI work? Genetic algorithm outline\\
Dynamic systems theory\\
Damage minimisation\\
How do I want the AI to manipulate the relationship between the body and movement?\\
Why do I want there to be a relationship between the two? - article Simon sent\\
How am I going to test the relationship?\\
How will the robot be measured?\\

\subsection{Extension Objectives}
\subsubsection{Vision}
How would a rudimentary visual system reduce damage to the robot?\\
How would this move the AI from a reactive to proactive mechanism?\\
\subsubsection{Terrain Friction}
How do snakes work with different frictions?\\
\subsubsection{UI}
How could a UI help lower the threshold to the project and make it easier to 'work with' the AI?\\
\subsubsection{Flexible Tail}
What are the advantages of having a flexible tail?\\


\newpage
\section{Project Relevance}
\subsection{Salamandra Robotica II}
Insert from interim report
\subsection{Agama Robot}
Insert from interim report
\subsection{tbc}
Find a team that have modelled a robot vs building one

\newpage
\section{Requirements Analysis}
Insert from interim report - needs some work\\
Add section on the constraints of this project

\newpage
\section{Professional and Ethical Considerations}
Insert from interim report with more reference to code of conduct

\newpage
\section{Implementation}
\subsection{Body}
Insert from progress log \& CPG log\\
The velocity that should be applied to this section is calculated using\\\\
\begin{Large}
$\overrightarrow{v_{i}} = v_{i-1} + \frac{m}{2}\overrightarrow{w} $\\\\
\end{Large}
For modules $i = 0, ..., m$, where $m \in n$, the value of $w$ will be calculated by alternating between $S$ and $C$.\\\\
\begin{Large}
$S: \overrightarrow{w} = sin\overrightarrow{\theta_{i-1}} + sin\overrightarrow{\theta_{i}}$\\\\
$C: \overrightarrow{w} = cos\overrightarrow{\theta_{i-1}} + cos\overrightarrow{\theta_{i}}$\\\\
\end{Large}
\subsection{Tail}
Insert from tail log\\
\begin{Large}\\
$L = \sum^{n}_{i} r_{i}m_{i}\overrightarrow{v_{i}}$\\\\
$\overrightarrow{v_{t}} = \frac{L}{r_{t}m_{t}}$
\end{Large}

\subsection{Legs}
\subsection{Terrain}
Insert from terrain log
\subsection{Performance}
Insert from trapped algorithm log


\newpage
\section{Results}

\newpage
\section{Conclusion}

\newpage
\section{References}

\newpage
\section{Appendices}
\subsection{Code of Conduct}





\end{document}